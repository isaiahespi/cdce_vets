% Options for packages loaded elsewhere
\PassOptionsToPackage{unicode}{hyperref}
\PassOptionsToPackage{hyphens}{url}
\PassOptionsToPackage{dvipsnames,svgnames,x11names}{xcolor}
%
\documentclass[
  11pt,
  a4paper,
]{article}

\usepackage{amsmath,amssymb}
\usepackage{setspace}
\usepackage{iftex}
\ifPDFTeX
  \usepackage[T1]{fontenc}
  \usepackage[utf8]{inputenc}
  \usepackage{textcomp} % provide euro and other symbols
\else % if luatex or xetex
  \usepackage{unicode-math}
  \defaultfontfeatures{Scale=MatchLowercase}
  \defaultfontfeatures[\rmfamily]{Ligatures=TeX,Scale=1}
\fi
\usepackage{lmodern}
\ifPDFTeX\else  
    % xetex/luatex font selection
    \setmainfont[]{Palatino Linotype}
\fi
% Use upquote if available, for straight quotes in verbatim environments
\IfFileExists{upquote.sty}{\usepackage{upquote}}{}
\IfFileExists{microtype.sty}{% use microtype if available
  \usepackage[]{microtype}
  \UseMicrotypeSet[protrusion]{basicmath} % disable protrusion for tt fonts
}{}
\makeatletter
\@ifundefined{KOMAClassName}{% if non-KOMA class
  \IfFileExists{parskip.sty}{%
    \usepackage{parskip}
  }{% else
    \setlength{\parindent}{0pt}
    \setlength{\parskip}{6pt plus 2pt minus 1pt}}
}{% if KOMA class
  \KOMAoptions{parskip=half}}
\makeatother
\usepackage{xcolor}
\usepackage[top=2.4cm,bottom=2.4cm,left=2.5cm,right=2.5cm]{geometry}
\setlength{\emergencystretch}{3em} % prevent overfull lines
\setcounter{secnumdepth}{-\maxdimen} % remove section numbering


\providecommand{\tightlist}{%
  \setlength{\itemsep}{0pt}\setlength{\parskip}{0pt}}\usepackage{longtable,booktabs,array}
\usepackage{calc} % for calculating minipage widths
% Correct order of tables after \paragraph or \subparagraph
\usepackage{etoolbox}
\makeatletter
\patchcmd\longtable{\par}{\if@noskipsec\mbox{}\fi\par}{}{}
\makeatother
% Allow footnotes in longtable head/foot
\IfFileExists{footnotehyper.sty}{\usepackage{footnotehyper}}{\usepackage{footnote}}
\makesavenoteenv{longtable}
\usepackage{graphicx}
\makeatletter
\def\maxwidth{\ifdim\Gin@nat@width>\linewidth\linewidth\else\Gin@nat@width\fi}
\def\maxheight{\ifdim\Gin@nat@height>\textheight\textheight\else\Gin@nat@height\fi}
\makeatother
% Scale images if necessary, so that they will not overflow the page
% margins by default, and it is still possible to overwrite the defaults
% using explicit options in \includegraphics[width, height, ...]{}
\setkeys{Gin}{width=\maxwidth,height=\maxheight,keepaspectratio}
% Set default figure placement to htbp
\makeatletter
\def\fps@figure{htbp}
\makeatother

\usepackage{booktabs}
\usepackage{longtable}
\usepackage{array}
\usepackage{multirow}
\usepackage{wrapfig}
\usepackage{float}
\usepackage{colortbl}
\usepackage{pdflscape}
\usepackage{tabu}
\usepackage{threeparttable}
\usepackage{threeparttablex}
\usepackage[normalem]{ulem}
\usepackage{makecell}
\usepackage{xcolor}
\makeatletter
\@ifpackageloaded{caption}{}{\usepackage{caption}}
\AtBeginDocument{%
\ifdefined\contentsname
  \renewcommand*\contentsname{Table of contents}
\else
  \newcommand\contentsname{Table of contents}
\fi
\ifdefined\listfigurename
  \renewcommand*\listfigurename{List of Figures}
\else
  \newcommand\listfigurename{List of Figures}
\fi
\ifdefined\listtablename
  \renewcommand*\listtablename{List of Tables}
\else
  \newcommand\listtablename{List of Tables}
\fi
\ifdefined\figurename
  \renewcommand*\figurename{Figure}
\else
  \newcommand\figurename{Figure}
\fi
\ifdefined\tablename
  \renewcommand*\tablename{Table}
\else
  \newcommand\tablename{Table}
\fi
}
\@ifpackageloaded{float}{}{\usepackage{float}}
\floatstyle{ruled}
\@ifundefined{c@chapter}{\newfloat{codelisting}{h}{lop}}{\newfloat{codelisting}{h}{lop}[chapter]}
\floatname{codelisting}{Listing}
\newcommand*\listoflistings{\listof{codelisting}{List of Listings}}
\makeatother
\makeatletter
\makeatother
\makeatletter
\@ifpackageloaded{caption}{}{\usepackage{caption}}
\@ifpackageloaded{subcaption}{}{\usepackage{subcaption}}
\makeatother

\ifLuaTeX
  \usepackage{selnolig}  % disable illegal ligatures
\fi
\usepackage{bookmark}

\IfFileExists{xurl.sty}{\usepackage{xurl}}{} % add URL line breaks if available
\urlstyle{same} % disable monospaced font for URLs
\hypersetup{
  pdftitle={Survey Experiment},
  pdfauthor={Isaiah},
  pdfkeywords={Election Workers, Poll workers, Veterans, Public
opinion, Election administration},
  colorlinks=true,
  linkcolor={blue},
  filecolor={Maroon},
  citecolor={Blue},
  urlcolor={Blue},
  pdfcreator={LaTeX via pandoc}}

%% CAPTIONS
\usepackage{caption}
\DeclareCaptionStyle{italic}[justification=centering]
 {labelfont={bf},textfont={it},labelsep=colon}
\captionsetup[figure]{style=italic,format=hang,singlelinecheck=true}
\captionsetup[table]{style=italic,format=hang,singlelinecheck=true}

%% FONT
% \usepackage{bera}
% \usepackage[charter]{mathdesign}
% \usepackage[scale=0.9]{sourcecodepro}
% \usepackage[lf,t]{FiraSans}
\usepackage{fontawesome}

%% HEADERS AND FOOTERS
\usepackage{fancyhdr}
\pagestyle{fancy}
\rfoot{\Large\sffamily\raisebox{-0.1cm}{\textbf{\thepage}}}
\makeatletter
\lhead{\textsf{\expandafter{\@title}}}
\makeatother
\rhead{}
\cfoot{}
\setlength{\headheight}{15pt}
\renewcommand{\headrulewidth}{0.4pt}
\renewcommand{\footrulewidth}{0.4pt}
\fancypagestyle{plain}{%
\fancyhf{} % clear all header and footer fields
\fancyfoot[C]{\sffamily\thepage} % except the center
\renewcommand{\headrulewidth}{0pt}
\renewcommand{\footrulewidth}{0pt}}

%% MATHS
\usepackage{bm,amsmath}
\allowdisplaybreaks

%% GRAPHICS
\makeatletter
\def\fps@figure{htbp}
\makeatother
\setcounter{topnumber}{2}
\setcounter{bottomnumber}{2}
\setcounter{totalnumber}{4}
\renewcommand{\topfraction}{0.85}
\renewcommand{\bottomfraction}{0.85}
\renewcommand{\textfraction}{0.15}
\renewcommand{\floatpagefraction}{0.8}

%% SECTION TITLES
\usepackage[compact,sf,bf]{titlesec}
\titleformat*{\section}{\Large\sf\bfseries\color[rgb]{0.7,0,0}}
\titleformat*{\subsection}{\large\sf\bfseries\color[rgb]{0.7,0,0}}
\titleformat*{\subsubsection}{\sf\bfseries\color[rgb]{0.7,0,0}}
\titlespacing{\section}{0pt}{2ex}{.5ex}
\titlespacing{\subsection}{0pt}{1.5ex}{0ex}
\titlespacing{\subsubsection}{0pt}{.5ex}{0ex}


%% BIBLIOGRAPHY.

\makeatletter
\@ifpackageloaded{biblatex}{
\ExecuteBibliographyOptions{bibencoding=utf8,minnames=1,maxnames=3, maxbibnames=99,dashed=false,terseinits=true,giveninits=true,uniquename=false,uniquelist=false,doi=false, isbn=false,url=true,sortcites=false}
\DeclareFieldFormat{url}{\texttt{\url{#1}}}
\DeclareFieldFormat[article]{pages}{#1}
\DeclareFieldFormat[inproceedings]{pages}{\lowercase{pp.}#1}
\DeclareFieldFormat[incollection]{pages}{\lowercase{pp.}#1}
\DeclareFieldFormat[article]{volume}{\mkbibbold{#1}}
\DeclareFieldFormat[article]{number}{\mkbibparens{#1}}
\DeclareFieldFormat[article]{title}{\MakeCapital{#1}}
\DeclareFieldFormat[article]{url}{}
\DeclareFieldFormat[inproceedings]{title}{#1}
\DeclareFieldFormat{shorthandwidth}{#1}
\usepackage{xpatch}
\xpatchbibmacro{volume+number+eid}{\setunit*{\adddot}}{}{}{}
% Remove In: for an article.
\renewbibmacro{in:}{%
  \ifentrytype{article}{}{%
  \printtext{\bibstring{in}\intitlepunct}}}
\AtEveryBibitem{\clearfield{month}}
\AtEveryCitekey{\clearfield{month}}
\DeclareDelimFormat[cbx@textcite]{nameyeardelim}{\addspace}
\renewcommand*{\finalnamedelim}{\addspace\&\space}
}{}
\makeatother

%% PAGE BREAKING to avoid widows and orphans
\clubpenalty = 2000
\widowpenalty = 2000
\usepackage{microtype}
% Placement of logos

\RequirePackage[absolute,overlay]{textpos}
\setlength{\TPHorizModule}{1cm}
\setlength{\TPVertModule}{1cm}
\def\placefig#1#2#3#4{\begin{textblock}{.1}(#1,#2)\rlap{\includegraphics[#3]{#4}}\end{textblock}}

% Title and date

\title{Survey Experiment}
\date{2024-9-12}

\def\Date{\number\day}
\def\Month{\ifcase\month\or
 January\or February\or March\or April\or May\or June\or
 July\or August\or September\or October\or November\or December\fi}
\def\Year{\number\year}

%%%% PAGE STYLE FOR FRONT PAGE OF REPORTS

\makeatletter
\def\organization#1{\gdef\@organization{#1}}
\def\telephone#1{\gdef\@telephone{#1}}
\def\email#1{\gdef\@email{#1}}
\makeatother
  \organization{}

  \def\name{Center for Democracy and Civic Engagement}
% 
  \telephone{(410) 903 6911}
% 
  \email{ssnovey@umd.edu}

\def\webaddress{\url{https://cdce.umd.edu/}}
\def\abn{12 377 614 012}
\def\extraspace{\vspace*{1.6cm}}
\makeatletter
\def\contactdetails{\faicon{phone} & \@telephone \\
                    \faicon{envelope} & \@email}
\makeatother

\usepackage[absolute,overlay]{textpos}
\setlength{\TPHorizModule}{1cm}
\setlength{\TPVertModule}{1cm}

%%%% FRONT PAGE OF REPORTS

% \def\reporttype{Report}

\long\def\front#1#2#3{
\newpage
% \begin{textblock}{7}(12.7,28.2)\hfill
% \includegraphics[height=4cm]{umd_logo2}~~~
% \end{textblock}
\begin{singlespacing}
\thispagestyle{empty}
\vspace*{-1.4cm}
\hspace*{-1.4cm}
\hbox to 16cm{
  \hbox to 6.5cm{\vbox to 14cm{\vbox to 25cm{
    \includegraphics[width=6.6cm]{CDCE}
    \vfill
    \includegraphics[width=5cm]{umd_logo2}
    \vspace{0.4cm}
    \par
    \parbox{6.3cm}{\raggedright
      \sf\color[rgb]{0.70,0.00,0.00}
      {\large\textbf{\name}}\par
      \vspace{.7cm}
      \tabcolsep=0.12cm\sf\small
      \begin{tabular}{@{}ll@{}}\contactdetails
      \end{tabular}
    }
  }\vss}\hss}
  \hspace*{0.2cm}
  \hbox to 1cm{\vbox to 14cm{\rule{1pt}{26.8cm}\vss}\hss\hfill}
  \hbox to 10cm{\vbox to 14cm{\vbox to 25cm{
      \vspace*{3cm}\sf\raggedright
      \parbox{10cm}{\sf\raggedright\baselineskip=1.2cm
         \fontsize{24.88}{30}\color[rgb]{0.70,0.00,0.00}\sf\textbf{#1}}
      \par
      \vfill
      \large
      \vbox{\parskip=0.8cm #2}\par
      \vspace*{2cm}\par
      % \reporttype\\[0.3cm]
      % \hbox{#3}%\\[2cm]\
      % \vspace*{1cm}
      %{\large\sf\textbf{\Date~\Month~\Year}}
   }\vss}
  }}
\end{singlespacing}
\newpage
}

\makeatletter
\def\maketitle{\front{\expandafter{\@title}}{\@author}{\@organization}}
\makeatother

% Authors

\author{\sf{\Large\textbf{Isaiah} \\[0.5cm]}}
%\lfoot{\sf Isaiah: 2024-9-12}
\begin{document}
\maketitle


\setstretch{1.5}
\subsection{Survey Questions by Treatment
Condition}\label{survey-questions-by-treatment-condition}

\subsubsection{Trust and Confidence in AZ
Elections}\label{trust-and-confidence-in-az-elections}

The following cross-tables break down relative observations and
percentages of responses to survey questions across categories of the
survey's experimental condition. The survey questions in the table below
concerned a respondent's degree of trust and confidence in the electoral
process in Maricopa County, AZ.

\begin{table}

\caption{\label{tbl-2}}

\centering{

Trust and Confidence in Maricopa County, AZ Elections by Treatment
Condition

~

Control

Treatment

P-value

(N=693)

(N=695)

Q19. Accurate counts, AZ

Not\_confident

142 (20.5\%)

112 (16.1\%)

0.0363

Confident

512 (73.9\%)

547 (78.7\%)

Missing

39 (5.6\%)

36 (5.2\%)

Q20. Competence of Election Staff, AZ

Not\_confident

136 (19.6\%)

112 (16.1\%)

0.0887

Confident

516 (74.5\%)

546 (78.6\%)

Missing

41 (5.9\%)

37 (5.3\%)

Q21. Commitment of Election Staff, AZ

not\_committed

116 (16.7\%)

101 (14.5\%)

0.255

committed

535 (77.2\%)

558 (80.3\%)

Missing

42 (6.1\%)

36 (5.2\%)

Q22. Fair Process, AZ

Not\_confident

155 (22.4\%)

114 (16.4\%)

0.00459

Confident

496 (71.6\%)

544 (78.3\%)

Missing

42 (6.1\%)

37 (5.3\%)

Q23. Fair Outcomes, AZ

Not\_confident

149 (21.5\%)

127 (18.3\%)

0.128

Confident

501 (72.3\%)

530 (76.3\%)

Missing

43 (6.2\%)

38 (5.5\%)

Q24. Secure Voting Tech, AZ

Not\_confident

199 (28.7\%)

160 (23.0\%)

0.0105

Confident

448 (64.6\%)

499 (71.8\%)

Missing

46 (6.6\%)

36 (5.2\%)

Q25. Voter Intimidation/Violence, AZ

not\_concerned

300 (43.3\%)

356 (51.2\%)

0.00616

concerned

347 (50.1\%)

302 (43.5\%)

Missing

46 (6.6\%)

37 (5.3\%)

Q26. Safe In-person Voting, AZ

Not\_confident

145 (20.9\%)

105 (15.1\%)

0.00382

Confident

501 (72.3\%)

552 (79.4\%)

Missing

47 (6.8\%)

38 (5.5\%)

Q27. Election Official Approval, AZ

disapprove

168 (24.2\%)

143 (20.6\%)

0.0861

approve

476 (68.7\%)

511 (73.5\%)

Missing

49 (7.1\%)

41 (5.9\%)

Q29. Adopt AZ Program

oppose

172 (24.8\%)

147 (21.2\%)

0.0688

support

465 (67.1\%)

507 (72.9\%)

Missing

56 (8.1\%)

41 (5.9\%)

{Note: }

Table reflects column percentages. P-values based on Pearson's
Chi-squared test of independence.

}

\end{table}%

The next table reviews a set of questions designed to assess a
respondent's expectation of electoral fraud of some sort in Maricopa
County, AZ. Again, responses are compared across categories of the
survey experiment.

\begin{table}

\caption{\label{tbl-3}}

\centering{

Expectation of Electoral Fraud in Maricopa County, AZ by Treatment
Condition

~

Control

Treatment

P-value

(N=693)

(N=695)

Q28\_1. Voter fraud, AZ

Not\_likely

344 (49.6\%)

382 (55.0\%)

0.116

Likely

295 (42.6\%)

273 (39.3\%)

Missing

54 (7.8\%)

40 (5.8\%)

Q28\_2. Votes won't be counted, AZ

Not\_likely

360 (51.9\%)

413 (59.4\%)

0.0133

Likely

279 (40.3\%)

240 (34.5\%)

Missing

54 (7.8\%)

42 (6.0\%)

Q28\_3. People will turned away, AZ

Not\_likely

326 (47.0\%)

378 (54.4\%)

0.0154

Likely

313 (45.2\%)

275 (39.6\%)

Missing

54 (7.8\%)

42 (6.0\%)

Q28\_4. Foreign interference with votes, AZ

Not\_likely

417 (60.2\%)

444 (63.9\%)

0.345

Likely

222 (32.0\%)

210 (30.2\%)

Missing

54 (7.8\%)

41 (5.9\%)

Q28\_5. EOs discourage people from voting, AZ

Not\_likely

387 (55.8\%)

447 (64.3\%)

0.00415

Likely

252 (36.4\%)

207 (29.8\%)

Missing

54 (7.8\%)

41 (5.9\%)

{Note: }

Table reflects column percentages. P-values based on Pearson's
Chi-squared test of independence.

}

\end{table}%

\subsection{Impact on confidence in fairness and
accuracy}\label{impact-on-confidence-in-fairness-and-accuracy}

The following examines survey questions that asked respondents whether
particular circumstances or election official actions would have any
impact on their confidence in the fairness and accuracy of elections. Of
particular importance is question number Q43\_4 which asked respondents
to consider whether inclusion of military veterans and veteran's family
members as election staff and volunteers has any impact on their
confidence in the fairness and accuracy of elections in November.

\begin{table}

\caption{\label{tbl-4}}

\centering{

Circumstantial Impact on Confidence in Fairness and Accuracy of
Elections by Treatment Condition

~

Control

Treatment

P-value

(N=693)

(N=695)

Q41\_1. Election officials test machines

decrease

37 (5.3\%)

21 (3.0\%)

0.069

no\_impact

67 (9.7\%)

77 (11.1\%)

increase

212 (30.6\%)

223 (32.1\%)

Missing

377 (54.4\%)

374 (53.8\%)

Q41\_2. Election officials conduct audits

decrease

39 (5.6\%)

22 (3.2\%)

0.0331

no\_impact

84 (12.1\%)

78 (11.2\%)

increase

193 (27.9\%)

221 (31.8\%)

Missing

377 (54.4\%)

374 (53.8\%)

Q41\_3. Partisan Poll watchers observe the election.

decrease

83 (12.0\%)

57 (8.2\%)

0.0187

no\_impact

122 (17.6\%)

125 (18.0\%)

increase

111 (16.0\%)

139 (20.0\%)

Missing

377 (54.4\%)

374 (53.8\%)

Q41\_4. Election staff include veterans and family

decrease

32 (4.6\%)

27 (3.9\%)

0.00556

no\_impact

138 (19.9\%)

105 (15.1\%)

increase

146 (21.1\%)

189 (27.2\%)

Missing

377 (54.4\%)

374 (53.8\%)

Q41\_5. Election staff include lawyers

decrease

47 (6.8\%)

27 (3.9\%)

0.0304

no\_impact

149 (21.5\%)

154 (22.2\%)

increase

120 (17.3\%)

140 (20.1\%)

Missing

377 (54.4\%)

374 (53.8\%)

Q41\_6. Election staff include college students

decrease

55 (7.9\%)

56 (8.1\%)

0.959

no\_impact

151 (21.8\%)

150 (21.6\%)

increase

110 (15.9\%)

115 (16.5\%)

Missing

377 (54.4\%)

374 (53.8\%)

{Note: }

Table reflects column percentages. P-values based on Pearson's
Chi-squared test of independence. Each question listed here was prefaced
with the following: Regardless of whether any of these are actually the
case, how would the following impact your confidence in the fairness and
accuracy of elections conducted this November?

}

\end{table}%

\subsection{Impact on concerns for voter
safety}\label{impact-on-concerns-for-voter-safety}

The following table examines a set of survey questions designed to
assess the impact of certain conditions or circumstances at election
sites on feelings of safety while voting in-person. Again, of particular
importance is question number Q43\_4 which has a respondent consider
whether knowledge that election staff includes veteran service members
and their family has any impact on their confidence that voters will be
safe from violence, threats, and intimidation while voting in person.

\begin{table}

\caption{\label{tbl-t5}}

\centering{

Circumstantial Impact on Confidence of voter safety by Treatment
Condition

~

Control

Treatment

P-value

(N=693)

(N=695)

Q43\_1. Law enforcement presence.

decrease

39 (5.6\%)

29 (4.2\%)

0.376

no\_impact

92 (13.3\%)

92 (13.2\%)

increase

184 (26.6\%)

198 (28.5\%)

Missing

378 (54.5\%)

376 (54.1\%)

Q43\_2. Partisan Poll watchers observe the election

decrease

77 (11.1\%)

63 (9.1\%)

0.33

no\_impact

132 (19.0\%)

137 (19.7\%)

increase

106 (15.3\%)

119 (17.1\%)

Missing

378 (54.5\%)

376 (54.1\%)

Q43\_3. People holding signs or giving out literature

decrease

99 (14.3\%)

85 (12.2\%)

0.349

no\_impact

155 (22.4\%)

162 (23.3\%)

increase

61 (8.8\%)

72 (10.4\%)

Missing

378 (54.5\%)

376 (54.1\%)

Q43\_4. Election staff includes veterans

decrease

40 (5.8\%)

29 (4.2\%)

0.151

no\_impact

116 (16.7\%)

107 (15.4\%)

increase

159 (22.9\%)

183 (26.3\%)

Missing

378 (54.5\%)

376 (54.1\%)

Q43\_5. Election staff includes lawyers

decrease

56 (8.1\%)

30 (4.3\%)

0.00872

no\_impact

155 (22.4\%)

172 (24.7\%)

increase

104 (15.0\%)

117 (16.8\%)

Missing

378 (54.5\%)

376 (54.1\%)

Q43\_6. Election staff includes students

decrease

66 (9.5\%)

53 (7.6\%)

0.359

no\_impact

149 (21.5\%)

162 (23.3\%)

increase

99 (14.3\%)

104 (15.0\%)

Missing

379 (54.7\%)

376 (54.1\%)

Table reflects column percentages. P-values based on Pearson's
Chi-squared test of independence. Each question listed here was prefaced
with the following: How would the following impact your confidence that
voters are safe from violence, threats of violence, or intimidation
while voting in-person during elections this November?

}

\end{table}%

\subsection{}\label{section}

\subsection{Q5. Attention Paid to Political
Affairs}\label{q5.-attention-paid-to-political-affairs}

\begin{table}

\caption{\label{tbl-q5q19}}

\centering{

\centering
\centering
\fontsize{12}{14}\selectfont
\begin{threeparttable}
\begin{tabular}[t]{lllll}
\toprule
\multicolumn{1}{l}{Treatment} & \multicolumn{1}{l}{Control} \\
\cmidrule(l{3pt}r{3pt}){1-1} \cmidrule(l{3pt}r{3pt}){2-2}
 & Q19. Vote count confidence, AZ &  &  & \\
\midrule
Q5 & Not\_confident & Confident & NA\_ & Total\\
Inattentive & 23.71  (78) & 68.39 (225) & 7.90 (26) & 100.00 (329)\\
Attentive & 17.58  (64) & 78.85 (287) & 3.57 (13) & 100.00 (364)\\
Total & 20.49 (142) & 73.88 (512) & 5.63 (39) & 100.00 (693)\\
\bottomrule
\end{tabular}
\begin{tablenotes}
\small
\item \textit{Note: } 
\item makecell[l]{Tables reflect row percentages\Q19. How confident are you that votes in Maricopa County, AZ will be counted as voters intend in the elections this November? \ Q5. How often do you pay attention to what is going on in government and politics?}
\end{tablenotes}
\end{threeparttable}
\centering
\begin{tabular}[t]{lllll}
\toprule
 & Q19. Vote count confidence, AZ &  &  & \\
\midrule
Q5 & Not\_confident & Confident & NA\_ & Total\\
Inattentive & 21.02  (62) & 72.54 (214) & 6.44 (19) & 100.00 (295)\\
Attentive & 12.50  (50) & 83.25 (333) & 4.25 (17) & 100.00 (400)\\
Total & 16.12 (112) & 78.71 (547) & 5.18 (36) & 100.00 (695)\\
\bottomrule
\end{tabular}

}

\end{table}%

\begin{table}

\caption{\label{tbl-q5q20}}

\centering{

\centering
\centering
\fontsize{12}{14}\selectfont
\begin{threeparttable}
\begin{tabular}[t]{l|l|l|l}
\hline
\multicolumn{1}{l|}{Treatment} & \multicolumn{1}{l}{Control} \\
\cline{1-1} \cline{2-2}
 & Q20. Competence of Election Staff, AZ &  & \\
\hline
Q5 & Not\_confident & Confident & NA\_\\
\hline
Inattentive & 21.58 & 70.21 & 8.21\\
\hline
Attentive & 17.86 & 78.30 & 3.85\\
\hline
\end{tabular}
\begin{tablenotes}
\small
\item \textit{Note: } 
\item makecell[l]{Tables reflect row percentages\Q20. How confident are you that election officials...in Maricopa County, AZ will do a good job conducting the elections this November? \ Q5. How often do you pay attention to what is going on in government and politics?}
\end{tablenotes}
\end{threeparttable}
\centering
\begin{tabular}[t]{l|l|l|l}
\hline
 & Q20. Competence of Election Staff, AZ &  & \\
\hline
Q5 & Not\_confident & Confident & NA\_\\
\hline
Inattentive & 19.32 & 74.24 & 6.44\\
\hline
Attentive & 13.75 & 81.75 & 4.50\\
\hline
\end{tabular}

}

\end{table}%

\begin{table}
\centering
\centering
\fontsize{12}{14}\selectfont
\begin{threeparttable}
\begin{tabular}[t]{l|l|l|l}
\hline
\multicolumn{1}{l|}{Treatment} & \multicolumn{1}{l}{Control} \\
\cline{1-1} \cline{2-2}
 & Q21. Commitment of Election Staff, AZ &  & \\
\hline
Q5 & not\_committed & committed & NA\_\\
\hline
Inattentive & 18.24 & 73.25 & 8.51\\
\hline
Attentive & 15.38 & 80.77 & 3.85\\
\hline
\end{tabular}
\begin{tablenotes}
\small
\item \textit{Note: } 
\item makecell[l]{Tables reflect row percentages\Q21. How committed do you think [Election staff in Maricopa County, AZ] will be to making sure the elections held this November are fair and accurate? \ Q5. How often do you pay attention to what is going on in government and politics?}
\end{tablenotes}
\end{threeparttable}
\centering
\begin{tabular}[t]{l|l|l|l}
\hline
 & Q21. Commitment of Election Staff, AZ &  & \\
\hline
Q5 & not\_committed & committed & NA\_\\
\hline
Inattentive & 15.93 & 77.63 & 6.44\\
\hline
Attentive & 13.50 & 82.25 & 4.25\\
\hline
\end{tabular}
\end{table}

\begin{table}
\centering
\centering
\begin{tabular}[t]{l|l|l|l}
\hline
\multicolumn{1}{l|}{Treatment} & \multicolumn{1}{l}{Control} \\
\cline{1-1} \cline{2-2}
 & Q22. Fair Process, AZ &  & \\
\hline
Q5 & Not\_confident & Confident & NA\_\\
\hline
Inattentive & 23.71 & 67.78 & 8.51\\
\hline
Attentive & 21.15 & 75.00 & 3.85\\
\hline
\multicolumn{4}{l}{\rule{0pt}{1em}\textit{Note: }}\\
\multicolumn{4}{l}{\rule{0pt}{1em}makecell[l]{Tables reflect row percentages\ Q22. How confident are you that the voting process will be fair in Maricopa County, AZ?\ Q5. How often do you pay attention to what is going on in government and politics?}}\\
\end{tabular}
\centering
\begin{tabular}[t]{l|l|l|l}
\hline
 & Q22. Fair Process, AZ &  & \\
\hline
Q5 & Not\_confident & Confident & NA\_\\
\hline
Inattentive & 19.66 & 73.56 & 6.78\\
\hline
Attentive & 14.00 & 81.75 & 4.25\\
\hline
\end{tabular}
\end{table}

\begin{table}
\centering
\centering
\begin{tabular}[t]{l|l|l|l}
\hline
\multicolumn{1}{l|}{Treatment} & \multicolumn{1}{l}{Control} \\
\cline{1-1} \cline{2-2}
 & Q23. Fair Outcomes, AZ &  & \\
\hline
Q5 & Not\_confident & Confident & NA\_\\
\hline
Inattentive & 21.88 & 69.30 & 8.81\\
\hline
Attentive & 21.15 & 75.00 & 3.85\\
\hline
\multicolumn{4}{l}{\rule{0pt}{1em}\textit{Note: }}\\
\multicolumn{4}{l}{\rule{0pt}{1em}makecell[l]{Tables reflect row percentages\ Q23. How confident are you that the voting outcomes will be fair in Maricopa County, AZ?\ Q5. How often do you pay attention to what is going on in government and politics?}}\\
\end{tabular}
\centering
\begin{tabular}[t]{l|l|l|l}
\hline
 & Q23. Fair Outcomes, AZ &  & \\
\hline
Q5 & Not\_confident & Confident & NA\_\\
\hline
Inattentive & 23.05 & 69.83 & 7.12\\
\hline
Attentive & 14.75 & 81.00 & 4.25\\
\hline
\end{tabular}
\end{table}

\begin{table}
\centering
\centering
\begin{tabular}[t]{l|l|l|l}
\hline
\multicolumn{1}{l|}{Treatment} & \multicolumn{1}{l}{Control} \\
\cline{1-1} \cline{2-2}
 & Q24. Secure Voting Tech, AZ &  & \\
\hline
Q5 & Not\_confident & Confident & NA\_\\
\hline
Inattentive & 27.05 & 63.53 & 9.42\\
\hline
Attentive & 30.22 & 65.66 & 4.12\\
\hline
\multicolumn{4}{l}{\rule{0pt}{1em}\textit{Note: }}\\
\multicolumn{4}{l}{\rule{0pt}{1em}makecell[l]{Tables reflect row percentages\ Q24. How confident are you that election systems in Maricopa County, AZ will be secure from hacking and other technological threats?\ Q5. How often do you pay attention to what is going on in government and politics?}}\\
\end{tabular}
\centering
\begin{tabular}[t]{l|l|l|l}
\hline
 & Q24. Secure Voting Tech, AZ &  & \\
\hline
Q5 & Not\_confident & Confident & NA\_\\
\hline
Inattentive & 25.76 & 67.46 & 6.78\\
\hline
Attentive & 21.00 & 75.00 & 4.00\\
\hline
\end{tabular}
\end{table}

\begin{table}
\centering
\centering
\begin{tabular}[t]{l|l|l|l}
\hline
\multicolumn{1}{l|}{Treatment} & \multicolumn{1}{l}{Control} \\
\cline{1-1} \cline{2-2}
 & Q25. Voter Intimidation/Violence, AZ &  & \\
\hline
Q5 & not\_concerned & concerned & NA\_\\
\hline
Inattentive & 42.86 & 47.72 & 9.42\\
\hline
Attentive & 43.68 & 52.20 & 4.12\\
\hline
\multicolumn{4}{l}{\rule{0pt}{1em}\textit{Note: }}\\
\multicolumn{4}{l}{\rule{0pt}{1em}makecell[l]{Tables reflect row percentages\ Q25. Thinking about Maricopa County, AZ, how concerned should voters feel about potential violence, threats of violence, or intimidation while voting in person at their local polling place?\ Q5. How often do you pay attention to what is going on in government and politics?}}\\
\end{tabular}
\centering
\begin{tabular}[t]{l|l|l|l}
\hline
 & Q25. Voter Intimidation/Violence, AZ &  & \\
\hline
Q5 & not\_concerned & concerned & NA\_\\
\hline
Inattentive & 50.51 & 42.71 & 6.78\\
\hline
Attentive & 51.75 & 44.00 & 4.25\\
\hline
\end{tabular}
\end{table}

\begin{table}
\centering
\centering
\begin{tabular}[t]{l|l|l|l}
\hline
\multicolumn{1}{l|}{Treatment} & \multicolumn{1}{l}{Control} \\
\cline{1-1} \cline{2-2}
 & Q26. Safe In-person Voting, AZ &  & \\
\hline
Q5 & Not\_confident & Confident & NA\_\\
\hline
Inattentive & 22.19 & 68.09 & 9.73\\
\hline
Attentive & 19.78 & 76.10 & 4.12\\
\hline
\multicolumn{4}{l}{\rule{0pt}{1em}\textit{Note: }}\\
\multicolumn{4}{l}{\rule{0pt}{1em}makecell[l]{Tables reflect row percentages\ Q26. How confident...are you that in person polling places in Maricopa County, AZ will be safe places for voters...in November?\ Q5. How often do you pay attention to what is going on in government and politics?}}\\
\end{tabular}
\centering
\begin{tabular}[t]{l|l|l|l}
\hline
 & Q26. Safe In-person Voting, AZ &  & \\
\hline
Q5 & Not\_confident & Confident & NA\_\\
\hline
Inattentive & 16.27 & 76.61 & 7.12\\
\hline
Attentive & 14.25 & 81.50 & 4.25\\
\hline
\end{tabular}
\end{table}

\begin{table}
\centering
\centering
\begin{tabular}[t]{l|l|l|l}
\hline
\multicolumn{1}{l|}{Treatment} & \multicolumn{1}{l}{Control} \\
\cline{1-1} \cline{2-2}
 & Q27. Election Official Approval, AZ &  & \\
\hline
Q5 & disapprove & approve & NA\_\\
\hline
Inattentive & 24.32 & 65.65 & 10.03\\
\hline
Attentive & 24.18 & 71.43 & 4.40\\
\hline
\multicolumn{4}{l}{\rule{0pt}{1em}\textit{Note: }}\\
\multicolumn{4}{l}{\rule{0pt}{1em}makecell[l]{Tables reflect row percentages\ Q27. Do you approve or disapprove of the way election officials in Maricopa County, AZ are handling their jobs?\ Q5. How often do you pay attention to what is going on in government and politics?}}\\
\end{tabular}
\centering
\begin{tabular}[t]{l|l|l|l}
\hline
 & Q27. Election Official Approval, AZ &  & \\
\hline
Q5 & disapprove & approve & NA\_\\
\hline
Inattentive & 22.37 & 70.85 & 6.78\\
\hline
Attentive & 19.25 & 75.50 & 5.25\\
\hline
\end{tabular}
\end{table}

\subsection{Q6}\label{q6}

\begin{table}

\caption{\label{tbl-q6q19}}

\centering{

\centering
\centering
\begin{tabular}[t]{l|l|l|l}
\hline
\multicolumn{1}{l|}{Treatment} & \multicolumn{1}{l}{Control} \\
\cline{1-1} \cline{2-2}
 & Q19. Vote count confidence, AZ &  & \\
\hline
Q6 & Not\_confident & Confident & NA\_\\
\hline
Low Favor & 32.95 & 61.27 & 5.78\\
\hline
Mid Favor & 21.34 & 71.54 & 7.11\\
\hline
High Favor & 11.61 & 84.27 & 4.12\\
\hline
\multicolumn{4}{l}{\rule{0pt}{1em}\textit{Note: }}\\
\multicolumn{4}{l}{\rule{0pt}{1em}makecell[l]{Tables reflect row percentages\Q19. How confident are you that votes in Maricopa County, AZ will be counted as voters intend in the elections this November? \ Q6. In general, how favorable or unfavorable is your impression of local election officials?}}\\
\end{tabular}
\centering
\begin{tabular}[t]{l|l|l|l}
\hline
 & Q19. Vote count confidence, AZ &  & \\
\hline
Q6 & Not\_confident & Confident & NA\_\\
\hline
Low Favor & 20.32 & 75.94 & 3.74\\
\hline
Mid Favor & 21.80 & 68.72 & 9.48\\
\hline
High Favor & 9.12 & 87.84 & 3.04\\
\hline
NA & 100.00 & 0.00 & 0.00\\
\hline
\end{tabular}

}

\end{table}%

\begin{table}

\caption{\label{tbl-q6q20}}

\centering{

\centering
\centering
\begin{tabular}[t]{l|l|l|l}
\hline
\multicolumn{1}{l|}{Treatment} & \multicolumn{1}{l}{Control} \\
\cline{1-1} \cline{2-2}
 & Q20. Competence of Election Staff, AZ &  & \\
\hline
Q6 & Not\_confident & Confident & NA\_\\
\hline
Low Favor & 30.06 & 63.58 & 6.36\\
\hline
Mid Favor & 20.16 & 72.73 & 7.11\\
\hline
High Favor & 12.36 & 83.15 & 4.49\\
\hline
\multicolumn{4}{l}{\rule{0pt}{1em}\textit{Note: }}\\
\multicolumn{4}{l}{\rule{0pt}{1em}makecell[l]{Tables reflect row percentages\Q20. How confident are you that election officials...in Maricopa County, AZ will do a good job conducting the elections this November? \ Q6. In general, how favorable or unfavorable is your impression of local election officials?}}\\
\end{tabular}
\centering
\begin{tabular}[t]{l|l|l|l}
\hline
 & Q20. Competence of Election Staff, AZ &  & \\
\hline
Q6 & Not\_confident & Confident & NA\_\\
\hline
Low Favor & 20.32 & 75.94 & 3.74\\
\hline
Mid Favor & 20.38 & 70.14 & 9.48\\
\hline
High Favor & 10.14 & 86.49 & 3.38\\
\hline
NA & 100.00 & 0.00 & 0.00\\
\hline
\end{tabular}

}

\end{table}%

\begin{table}
\centering
\centering
\begin{tabular}[t]{l|l|l|l}
\hline
\multicolumn{1}{l|}{Treatment} & \multicolumn{1}{l}{Control} \\
\cline{1-1} \cline{2-2}
 & Q21. Commitment of Election Staff, AZ &  & \\
\hline
Q6 & not\_committed & committed & NA\_\\
\hline
Low Favor & 22.54 & 71.10 & 6.36\\
\hline
Mid Favor & 17.00 & 75.49 & 7.51\\
\hline
High Favor & 12.73 & 82.77 & 4.49\\
\hline
\multicolumn{4}{l}{\rule{0pt}{1em}\textit{Note: }}\\
\multicolumn{4}{l}{\rule{0pt}{1em}makecell[l]{Tables reflect row percentages\Q21. How committed do you think [Election staff in Maricopa County, AZ] will be to making sure the elections held this November are fair and accurate? \ Q6. In general, how favorable or unfavorable is your impression of local election officials?}}\\
\end{tabular}
\centering
\begin{tabular}[t]{l|l|l|l}
\hline
 & Q21. Commitment of Election Staff, AZ &  & \\
\hline
Q6 & not\_committed & committed & NA\_\\
\hline
Low Favor & 17.65 & 78.61 & 3.74\\
\hline
Mid Favor & 18.01 & 72.51 & 9.48\\
\hline
High Favor & 9.80 & 87.16 & 3.04\\
\hline
NA & 100.00 & 0.00 & 0.00\\
\hline
\end{tabular}
\end{table}

\begin{table}
\centering
\centering
\begin{tabular}[t]{l|l|l|l}
\hline
\multicolumn{1}{l|}{Treatment} & \multicolumn{1}{l}{Control} \\
\cline{1-1} \cline{2-2}
 & Q22. Fair Process, AZ &  & \\
\hline
Q6 & Not\_confident & Confident & NA\_\\
\hline
Low Favor & 31.21 & 62.43 & 6.36\\
\hline
Mid Favor & 23.72 & 68.77 & 7.51\\
\hline
High Favor & 15.36 & 80.15 & 4.49\\
\hline
\multicolumn{4}{l}{\rule{0pt}{1em}\textit{Note: }}\\
\multicolumn{4}{l}{\rule{0pt}{1em}makecell[l]{Tables reflect row percentages\ Q22. How confident are you that the voting process will be fair in Maricopa County, AZ?\ Q6. In general, how favorable or unfavorable is your impression of local election officials?}}\\
\end{tabular}
\centering
\begin{tabular}[t]{l|l|l|l}
\hline
 & Q22. Fair Process, AZ &  & \\
\hline
Q6 & Not\_confident & Confident & NA\_\\
\hline
Low Favor & 19.79 & 76.47 & 3.74\\
\hline
Mid Favor & 19.91 & 70.14 & 9.95\\
\hline
High Favor & 11.49 & 85.47 & 3.04\\
\hline
NA & 100.00 & 0.00 & 0.00\\
\hline
\end{tabular}
\end{table}

\begin{table}
\centering
\centering
\begin{tabular}[t]{l|l|l|l}
\hline
\multicolumn{1}{l|}{Treatment} & \multicolumn{1}{l}{Control} \\
\cline{1-1} \cline{2-2}
 & Q23. Fair Outcomes, AZ &  & \\
\hline
Q6 & Not\_confident & Confident & NA\_\\
\hline
Low Favor & 30.06 & 63.01 & 6.94\\
\hline
Mid Favor & 23.72 & 68.77 & 7.51\\
\hline
High Favor & 13.86 & 81.65 & 4.49\\
\hline
\multicolumn{4}{l}{\rule{0pt}{1em}\textit{Note: }}\\
\multicolumn{4}{l}{\rule{0pt}{1em}makecell[l]{Tables reflect row percentages\ Q23. How confident are you that the voting outcomes will be fair in Maricopa County, AZ?\ Q6. In general, how favorable or unfavorable is your impression of local election officials?}}\\
\end{tabular}
\centering
\begin{tabular}[t]{l|l|l|l}
\hline
 & Q23. Fair Outcomes, AZ &  & \\
\hline
Q6 & Not\_confident & Confident & NA\_\\
\hline
Low Favor & 22.99 & 73.26 & 3.74\\
\hline
Mid Favor & 21.80 & 67.77 & 10.43\\
\hline
High Favor & 12.50 & 84.46 & 3.04\\
\hline
NA & 100.00 & 0.00 & 0.00\\
\hline
\end{tabular}
\end{table}

\begin{table}
\centering
\centering
\begin{tabular}[t]{l|l|l|l}
\hline
\multicolumn{1}{l|}{Treatment} & \multicolumn{1}{l}{Control} \\
\cline{1-1} \cline{2-2}
 & Q24. Secure Voting Tech, AZ &  & \\
\hline
Q6 & Not\_confident & Confident & NA\_\\
\hline
Low Favor & 42.77 & 50.29 & 6.94\\
\hline
Mid Favor & 27.27 & 64.43 & 8.30\\
\hline
High Favor & 20.97 & 74.16 & 4.87\\
\hline
\multicolumn{4}{l}{\rule{0pt}{1em}\textit{Note: }}\\
\multicolumn{4}{l}{\rule{0pt}{1em}makecell[l]{Tables reflect row percentages\ Q24. How confident are you that election systems in Maricopa County, AZ will be secure from hacking and other technological threats?\ Q6. In general, how favorable or unfavorable is your impression of local election officials?}}\\
\end{tabular}
\centering
\begin{tabular}[t]{l|l|l|l}
\hline
 & Q24. Secure Voting Tech, AZ &  & \\
\hline
Q6 & Not\_confident & Confident & NA\_\\
\hline
Low Favor & 33.16 & 63.10 & 3.74\\
\hline
Mid Favor & 24.64 & 65.40 & 9.95\\
\hline
High Favor & 15.20 & 82.09 & 2.70\\
\hline
NA & 100.00 & 0.00 & 0.00\\
\hline
\end{tabular}
\end{table}

\begin{table}
\centering
\centering
\begin{tabular}[t]{l|l|l|l}
\hline
\multicolumn{1}{l|}{Treatment} & \multicolumn{1}{l}{Control} \\
\cline{1-1} \cline{2-2}
 & Q25. Voter Intimidation/Violence, AZ &  & \\
\hline
Q6 & not\_concerned & concerned & NA\_\\
\hline
Low Favor & 39.31 & 53.76 & 6.94\\
\hline
Mid Favor & 42.69 & 49.01 & 8.30\\
\hline
High Favor & 46.44 & 48.69 & 4.87\\
\hline
\multicolumn{4}{l}{\rule{0pt}{1em}\textit{Note: }}\\
\multicolumn{4}{l}{\rule{0pt}{1em}makecell[l]{Tables reflect row percentages\ Q25. Thinking about Maricopa County, AZ, how concerned should voters feel about potential violence, threats of violence, or intimidation while voting in person at their local polling place?\ Q6. In general, how favorable or unfavorable is your impression of local election officials?}}\\
\end{tabular}
\centering
\begin{tabular}[t]{l|l|l|l}
\hline
 & Q25. Voter Intimidation/Violence, AZ &  & \\
\hline
Q6 & not\_concerned & concerned & NA\_\\
\hline
Low Favor & 50.27 & 45.99 & 3.74\\
\hline
Mid Favor & 50.24 & 39.81 & 9.95\\
\hline
High Favor & 52.70 & 44.26 & 3.04\\
\hline
NA & 0.00 & 100.00 & 0.00\\
\hline
\end{tabular}
\end{table}

\begin{table}
\centering
\centering
\begin{tabular}[t]{l|l|l|l}
\hline
\multicolumn{1}{l|}{Treatment} & \multicolumn{1}{l}{Control} \\
\cline{1-1} \cline{2-2}
 & Q26. Safe In-person Voting, AZ &  & \\
\hline
Q6 & Not\_confident & Confident & NA\_\\
\hline
Low Favor & 27.75 & 65.32 & 6.94\\
\hline
Mid Favor & 20.95 & 70.36 & 8.70\\
\hline
High Favor & 16.48 & 78.65 & 4.87\\
\hline
\multicolumn{4}{l}{\rule{0pt}{1em}\textit{Note: }}\\
\multicolumn{4}{l}{\rule{0pt}{1em}makecell[l]{Tables reflect row percentages\ Q26. How confident...are you that in person polling places in Maricopa County, AZ will be safe places for voters...in November?\ Q6. In general, how favorable or unfavorable is your impression of local election officials?}}\\
\end{tabular}
\centering
\begin{tabular}[t]{l|l|l|l}
\hline
 & Q26. Safe In-person Voting, AZ &  & \\
\hline
Q6 & Not\_confident & Confident & NA\_\\
\hline
Low Favor & 22.46 & 73.80 & 3.74\\
\hline
Mid Favor & 17.54 & 72.04 & 10.43\\
\hline
High Favor & 8.45 & 88.51 & 3.04\\
\hline
NA & 100.00 & 0.00 & 0.00\\
\hline
\end{tabular}
\end{table}

\begin{table}
\centering
\centering
\begin{tabular}[t]{l|l|l|l}
\hline
\multicolumn{1}{l|}{Treatment} & \multicolumn{1}{l}{Control} \\
\cline{1-1} \cline{2-2}
 & Q27. Election Official Approval, AZ &  & \\
\hline
Q6 & disapprove & approve & NA\_\\
\hline
Low Favor & 35.84 & 57.23 & 6.94\\
\hline
Mid Favor & 22.13 & 68.77 & 9.09\\
\hline
High Favor & 18.73 & 76.03 & 5.24\\
\hline
\multicolumn{4}{l}{\rule{0pt}{1em}\textit{Note: }}\\
\multicolumn{4}{l}{\rule{0pt}{1em}makecell[l]{Tables reflect row percentages\ Q27. Do you approve or disapprove of the way election officials in Maricopa County, AZ are handling their jobs?\ Q6. In general, how favorable or unfavorable is your impression of local election officials?}}\\
\end{tabular}
\centering
\begin{tabular}[t]{l|l|l|l}
\hline
 & Q27. Election Official Approval, AZ &  & \\
\hline
Q6 & disapprove & approve & NA\_\\
\hline
Low Favor & 26.20 & 68.45 & 5.35\\
\hline
Mid Favor & 22.75 & 66.82 & 10.43\\
\hline
High Favor & 15.20 & 81.76 & 3.04\\
\hline
NA & 100.00 & 0.00 & 0.00\\
\hline
\end{tabular}
\end{table}

\subsection{Q7}\label{q7}

\begin{table}

\caption{\label{tbl-q7q19}}

\centering{

\centering
\centering
\begin{tabular}[t]{l|l|l|l}
\hline
\multicolumn{1}{l|}{Treatment} & \multicolumn{1}{l}{Control} \\
\cline{1-1} \cline{2-2}
 & Q19. Vote count confidence, AZ &  & \\
\hline
Q7 & Not\_confident & Confident & NA\_\\
\hline
Legitimate & 11.06 & 82.98 & 5.96\\
\hline
Not legitimate & 40.36 & 54.71 & 4.93\\
\hline
\multicolumn{4}{l}{\rule{0pt}{1em}\textit{Note: }}\\
\multicolumn{4}{l}{\rule{0pt}{1em}makecell[l]{Tables reflect row percentages\Q19. How confident are you that votes in Maricopa County, AZ will be counted as voters intend in the elections this November? \ Q7. Regardless of whom you supported in the 2020 election, do you think Joe Biden's election as president was legitimate, or was he not legitimately elected?}}\\
\end{tabular}
\centering
\begin{tabular}[t]{l|l|l|l}
\hline
 & Q19. Vote count confidence, AZ &  & \\
\hline
Q7 & Not\_confident & Confident & NA\_\\
\hline
Legitimate & 12.07 & 82.76 & 5.17\\
\hline
Not legitimate & 23.89 & 70.80 & 5.31\\
\hline
NA & 40.00 & 60.00 & 0.00\\
\hline
\end{tabular}

}

\end{table}%

\begin{table}

\caption{\label{tbl-q7q20}}

\centering{

\centering
\centering
\begin{tabular}[t]{l|l|l|l}
\hline
\multicolumn{1}{l|}{Treatment} & \multicolumn{1}{l}{Control} \\
\cline{1-1} \cline{2-2}
 & Q20. Competence of Election Staff, AZ &  & \\
\hline
Q7 & Not\_confident & Confident & NA\_\\
\hline
Legitimate & 11.06 & 82.77 & 6.17\\
\hline
Not legitimate & 37.67 & 56.95 & 5.38\\
\hline
\multicolumn{4}{l}{\rule{0pt}{1em}\textit{Note: }}\\
\multicolumn{4}{l}{\rule{0pt}{1em}makecell[l]{Tables reflect row percentages\Q20. How confident are you that election officials...in Maricopa County, AZ will do a good job conducting the elections this November? \ Q7. Regardless of whom you supported in the 2020 election, do you think Joe Biden's election as president was legitimate, or was he not legitimately elected?}}\\
\end{tabular}
\centering
\begin{tabular}[t]{l|l|l|l}
\hline
 & Q20. Competence of Election Staff, AZ &  & \\
\hline
Q7 & Not\_confident & Confident & NA\_\\
\hline
Legitimate & 11.64 & 82.97 & 5.39\\
\hline
Not legitimate & 24.78 & 69.91 & 5.31\\
\hline
NA & 40.00 & 60.00 & 0.00\\
\hline
\end{tabular}

}

\end{table}%

\begin{table}
\centering
\centering
\begin{tabular}[t]{l|l|l|l}
\hline
\multicolumn{1}{l|}{Treatment} & \multicolumn{1}{l}{Control} \\
\cline{1-1} \cline{2-2}
 & Q21. Commitment of Election Staff, AZ &  & \\
\hline
Q7 & not\_committed & committed & NA\_\\
\hline
Legitimate & 10.43 & 83.19 & 6.38\\
\hline
Not legitimate & 30.04 & 64.57 & 5.38\\
\hline
\multicolumn{4}{l}{\rule{0pt}{1em}\textit{Note: }}\\
\multicolumn{4}{l}{\rule{0pt}{1em}makecell[l]{Tables reflect row percentages\Q21. How committed do you think [Election staff in Maricopa County, AZ] will be to making sure the elections held this November are fair and accurate? \ Q7. Regardless of whom you supported in the 2020 election, do you think Joe Biden's election as president was legitimate, or was he not legitimately elected?}}\\
\end{tabular}
\centering
\begin{tabular}[t]{l|l|l|l}
\hline
 & Q21. Commitment of Election Staff, AZ &  & \\
\hline
Q7 & not\_committed & committed & NA\_\\
\hline
Legitimate & 11.64 & 83.19 & 5.17\\
\hline
Not legitimate & 19.91 & 74.78 & 5.31\\
\hline
NA & 40.00 & 60.00 & 0.00\\
\hline
\end{tabular}
\end{table}

\begin{table}
\centering
\centering
\begin{tabular}[t]{l|l|l|l}
\hline
\multicolumn{1}{l|}{Treatment} & \multicolumn{1}{l}{Control} \\
\cline{1-1} \cline{2-2}
 & Q22. Fair Process, AZ &  & \\
\hline
Q7 & Not\_confident & Confident & NA\_\\
\hline
Legitimate & 13.40 & 80.21 & 6.38\\
\hline
Not legitimate & 41.26 & 53.36 & 5.38\\
\hline
\multicolumn{4}{l}{\rule{0pt}{1em}\textit{Note: }}\\
\multicolumn{4}{l}{\rule{0pt}{1em}makecell[l]{Tables reflect row percentages\ Q22. How confident are you that the voting process will be fair in Maricopa County, AZ?\ Q7. Regardless of whom you supported in the 2020 election, do you think Joe Biden's election as president was legitimate, or was he not legitimately elected?}}\\
\end{tabular}
\centering
\begin{tabular}[t]{l|l|l|l}
\hline
 & Q22. Fair Process, AZ &  & \\
\hline
Q7 & Not\_confident & Confident & NA\_\\
\hline
Legitimate & 11.21 & 83.41 & 5.39\\
\hline
Not legitimate & 26.55 & 68.14 & 5.31\\
\hline
NA & 40.00 & 60.00 & 0.00\\
\hline
\end{tabular}
\end{table}

\begin{table}
\centering
\centering
\begin{tabular}[t]{l|l|l|l}
\hline
\multicolumn{1}{l|}{Treatment} & \multicolumn{1}{l}{Control} \\
\cline{1-1} \cline{2-2}
 & Q23. Fair Outcomes, AZ &  & \\
\hline
Q7 & Not\_confident & Confident & NA\_\\
\hline
Legitimate & 10.85 & 82.55 & 6.60\\
\hline
Not legitimate & 43.95 & 50.67 & 5.38\\
\hline
\multicolumn{4}{l}{\rule{0pt}{1em}\textit{Note: }}\\
\multicolumn{4}{l}{\rule{0pt}{1em}makecell[l]{Tables reflect row percentages\ Q23. How confident are you that the voting outcomes will be fair in Maricopa County, AZ?\ Q7. Regardless of whom you supported in the 2020 election, do you think Joe Biden's election as president was legitimate, or was he not legitimately elected?}}\\
\end{tabular}
\centering
\begin{tabular}[t]{l|l|l|l}
\hline
 & Q23. Fair Outcomes, AZ &  & \\
\hline
Q7 & Not\_confident & Confident & NA\_\\
\hline
Legitimate & 13.15 & 81.47 & 5.39\\
\hline
Not legitimate & 28.32 & 65.93 & 5.75\\
\hline
NA & 40.00 & 60.00 & 0.00\\
\hline
\end{tabular}
\end{table}

\begin{table}
\centering
\centering
\begin{tabular}[t]{l|l|l|l}
\hline
\multicolumn{1}{l|}{Treatment} & \multicolumn{1}{l}{Control} \\
\cline{1-1} \cline{2-2}
 & Q24. Secure Voting Tech, AZ &  & \\
\hline
Q7 & Not\_confident & Confident & NA\_\\
\hline
Legitimate & 17.87 & 75.11 & 7.02\\
\hline
Not legitimate & 51.57 & 42.60 & 5.83\\
\hline
\multicolumn{4}{l}{\rule{0pt}{1em}\textit{Note: }}\\
\multicolumn{4}{l}{\rule{0pt}{1em}makecell[l]{Tables reflect row percentages\ Q24. How confident are you that election systems in Maricopa County, AZ will be secure from hacking and other technological threats?\ Q7. Regardless of whom you supported in the 2020 election, do you think Joe Biden's election as president was legitimate, or was he not legitimately elected?}}\\
\end{tabular}
\centering
\begin{tabular}[t]{l|l|l|l}
\hline
 & Q24. Secure Voting Tech, AZ &  & \\
\hline
Q7 & Not\_confident & Confident & NA\_\\
\hline
Legitimate & 14.01 & 80.82 & 5.17\\
\hline
Not legitimate & 41.15 & 53.54 & 5.31\\
\hline
NA & 40.00 & 60.00 & 0.00\\
\hline
\end{tabular}
\end{table}

\begin{table}
\centering
\centering
\begin{tabular}[t]{l|l|l|l}
\hline
\multicolumn{1}{l|}{Treatment} & \multicolumn{1}{l}{Control} \\
\cline{1-1} \cline{2-2}
 & Q25. Voter Intimidation/Violence, AZ &  & \\
\hline
Q7 & not\_concerned & concerned & NA\_\\
\hline
Legitimate & 41.91 & 51.28 & 6.81\\
\hline
Not legitimate & 46.19 & 47.53 & 6.28\\
\hline
\multicolumn{4}{l}{\rule{0pt}{1em}\textit{Note: }}\\
\multicolumn{4}{l}{\rule{0pt}{1em}makecell[l]{Tables reflect row percentages\ Q25. Thinking about Maricopa County, AZ, how concerned should voters feel about potential violence, threats of violence, or intimidation while voting in person at their local polling place?\ Q7. Regardless of whom you supported in the 2020 election, do you think Joe Biden's election as president was legitimate, or was he not legitimately elected?}}\\
\end{tabular}
\centering
\begin{tabular}[t]{l|l|l|l}
\hline
 & Q25. Voter Intimidation/Violence, AZ &  & \\
\hline
Q7 & not\_concerned & concerned & NA\_\\
\hline
Legitimate & 50.65 & 43.97 & 5.39\\
\hline
Not legitimate & 52.65 & 42.04 & 5.31\\
\hline
NA & 40.00 & 60.00 & 0.00\\
\hline
\end{tabular}
\end{table}

\begin{table}
\centering
\centering
\begin{tabular}[t]{l|l|l|l}
\hline
\multicolumn{1}{l|}{Treatment} & \multicolumn{1}{l}{Control} \\
\cline{1-1} \cline{2-2}
 & Q26. Safe In-person Voting, AZ &  & \\
\hline
Q7 & Not\_confident & Confident & NA\_\\
\hline
Legitimate & 15.74 & 77.45 & 6.81\\
\hline
Not legitimate & 31.84 & 61.43 & 6.73\\
\hline
\multicolumn{4}{l}{\rule{0pt}{1em}\textit{Note: }}\\
\multicolumn{4}{l}{\rule{0pt}{1em}makecell[l]{Tables reflect row percentages\ Q26. How confident...are you that in person polling places in Maricopa County, AZ will be safe places for voters...in November?\ Q7. Regardless of whom you supported in the 2020 election, do you think Joe Biden's election as president was legitimate, or was he not legitimately elected?}}\\
\end{tabular}
\centering
\begin{tabular}[t]{l|l|l|l}
\hline
 & Q26. Safe In-person Voting, AZ &  & \\
\hline
Q7 & Not\_confident & Confident & NA\_\\
\hline
Legitimate & 11.85 & 82.76 & 5.39\\
\hline
Not legitimate & 21.24 & 73.01 & 5.75\\
\hline
NA & 40.00 & 60.00 & 0.00\\
\hline
\end{tabular}
\end{table}

\begin{table}
\centering
\centering
\begin{tabular}[t]{l|l|l|l}
\hline
\multicolumn{1}{l|}{Treatment} & \multicolumn{1}{l}{Control} \\
\cline{1-1} \cline{2-2}
 & Q27. Election Official Approval, AZ &  & \\
\hline
Q7 & disapprove & approve & NA\_\\
\hline
Legitimate & 15.96 & 76.81 & 7.23\\
\hline
Not legitimate & 41.70 & 51.57 & 6.73\\
\hline
\multicolumn{4}{l}{\rule{0pt}{1em}\textit{Note: }}\\
\multicolumn{4}{l}{\rule{0pt}{1em}makecell[l]{Tables reflect row percentages\ Q27. Do you approve or disapprove of the way election officials in Maricopa County, AZ are handling their jobs?\ Q7. Regardless of whom you supported in the 2020 election, do you think Joe Biden's election as president was legitimate, or was he not legitimately elected?}}\\
\end{tabular}
\centering
\begin{tabular}[t]{l|l|l|l}
\hline
 & Q27. Election Official Approval, AZ &  & \\
\hline
Q7 & disapprove & approve & NA\_\\
\hline
Legitimate & 16.59 & 77.80 & 5.60\\
\hline
Not legitimate & 28.32 & 65.93 & 5.75\\
\hline
NA & 40.00 & 20.00 & 40.00\\
\hline
\end{tabular}
\end{table}

\subsection{Q8}\label{q8}

\begin{table}

\caption{\label{tbl-q8q19}}

\centering{

\centering
\centering
\begin{tabular}[t]{l|l|l|l}
\hline
\multicolumn{1}{l|}{Treatment} & \multicolumn{1}{l}{Control} \\
\cline{1-1} \cline{2-2}
 & Q19. Vote count confidence, AZ &  & \\
\hline
Q8 & Not\_confident & Confident & NA\_\\
\hline
Low Trust & 30.86 & 62.89 & 6.25\\
\hline
Moderate Trust & 22.73 & 70.91 & 6.36\\
\hline
High Trust & 5.99 & 89.86 & 4.15\\
\hline
\multicolumn{4}{l}{\rule{0pt}{1em}\textit{Note: }}\\
\multicolumn{4}{l}{\rule{0pt}{1em}makecell[l]{Tables reflect row percentages\Q19. How confident are you that votes in Maricopa County, AZ will be counted as voters intend in the elections this November? \ Q8. Generally speaking, how often can you trust other people?}}\\
\end{tabular}
\centering
\begin{tabular}[t]{l|l|l|l}
\hline
 & Q19. Vote count confidence, AZ &  & \\
\hline
Q8 & Not\_confident & Confident & NA\_\\
\hline
Low Trust & 24.14 & 70.88 & 4.98\\
\hline
Moderate Trust & 15.58 & 80.90 & 3.52\\
\hline
High Trust & 7.66 & 85.53 & 6.81\\
\hline
\end{tabular}

}

\end{table}%

\begin{table}

\caption{\label{tbl-q8q20}}

\centering{

\centering
\centering
\begin{tabular}[t]{l|l|l|l}
\hline
\multicolumn{1}{l|}{Treatment} & \multicolumn{1}{l}{Control} \\
\cline{1-1} \cline{2-2}
 & Q20. Competence of Election Staff, AZ &  & \\
\hline
Q8 & Not\_confident & Confident & NA\_\\
\hline
Low Trust & 30.08 & 62.89 & 7.03\\
\hline
Moderate Trust & 20.45 & 73.18 & 6.36\\
\hline
High Trust & 6.45 & 89.40 & 4.15\\
\hline
\multicolumn{4}{l}{\rule{0pt}{1em}\textit{Note: }}\\
\multicolumn{4}{l}{\rule{0pt}{1em}makecell[l]{Tables reflect row percentages\Q20. How confident are you that election officials...in Maricopa County, AZ will do a good job conducting the elections this November? \ Q8. Generally speaking, how often can you trust other people?}}\\
\end{tabular}
\centering
\begin{tabular}[t]{l|l|l|l}
\hline
 & Q20. Competence of Election Staff, AZ &  & \\
\hline
Q8 & Not\_confident & Confident & NA\_\\
\hline
Low Trust & 24.14 & 70.88 & 4.98\\
\hline
Moderate Trust & 16.58 & 79.90 & 3.52\\
\hline
High Trust & 6.81 & 85.96 & 7.23\\
\hline
\end{tabular}

}

\end{table}%

\begin{table}
\centering
\centering
\begin{tabular}[t]{l|l|l|l}
\hline
\multicolumn{1}{l|}{Treatment} & \multicolumn{1}{l}{Control} \\
\cline{1-1} \cline{2-2}
 & Q21. Commitment of Election Staff, AZ &  & \\
\hline
Q8 & not\_committed & committed & NA\_\\
\hline
Low Trust & 22.66 & 70.31 & 7.03\\
\hline
Moderate Trust & 16.36 & 76.82 & 6.82\\
\hline
High Trust & 10.14 & 85.71 & 4.15\\
\hline
\multicolumn{4}{l}{\rule{0pt}{1em}\textit{Note: }}\\
\multicolumn{4}{l}{\rule{0pt}{1em}makecell[l]{Tables reflect row percentages\Q21. How committed do you think [Election staff in Maricopa County, AZ] will be to making sure the elections held this November are fair and accurate? \ Q8. Generally speaking, how often can you trust other people?}}\\
\end{tabular}
\centering
\begin{tabular}[t]{l|l|l|l}
\hline
 & Q21. Commitment of Election Staff, AZ &  & \\
\hline
Q8 & not\_committed & committed & NA\_\\
\hline
Low Trust & 20.69 & 74.33 & 4.98\\
\hline
Moderate Trust & 16.08 & 80.40 & 3.52\\
\hline
High Trust & 6.38 & 86.81 & 6.81\\
\hline
\end{tabular}
\end{table}

\begin{table}
\centering
\centering
\begin{tabular}[t]{l|l|l|l}
\hline
\multicolumn{1}{l|}{Treatment} & \multicolumn{1}{l}{Control} \\
\cline{1-1} \cline{2-2}
 & Q22. Fair Process, AZ &  & \\
\hline
Q8 & Not\_confident & Confident & NA\_\\
\hline
Low Trust & 32.03 & 60.94 & 7.03\\
\hline
Moderate Trust & 21.82 & 71.36 & 6.82\\
\hline
High Trust & 11.52 & 84.33 & 4.15\\
\hline
\multicolumn{4}{l}{\rule{0pt}{1em}\textit{Note: }}\\
\multicolumn{4}{l}{\rule{0pt}{1em}makecell[l]{Tables reflect row percentages\ Q22. How confident are you that the voting process will be fair in Maricopa County, AZ?\ Q8. Generally speaking, how often can you trust other people?}}\\
\end{tabular}
\centering
\begin{tabular}[t]{l|l|l|l}
\hline
 & Q22. Fair Process, AZ &  & \\
\hline
Q8 & Not\_confident & Confident & NA\_\\
\hline
Low Trust & 24.14 & 70.88 & 4.98\\
\hline
Moderate Trust & 19.10 & 76.88 & 4.02\\
\hline
High Trust & 5.53 & 87.66 & 6.81\\
\hline
\end{tabular}
\end{table}

\begin{table}
\centering
\centering
\begin{tabular}[t]{l|l|l|l}
\hline
\multicolumn{1}{l|}{Treatment} & \multicolumn{1}{l}{Control} \\
\cline{1-1} \cline{2-2}
 & Q23. Fair Outcomes, AZ &  & \\
\hline
Q8 & Not\_confident & Confident & NA\_\\
\hline
Low Trust & 30.08 & 62.89 & 7.03\\
\hline
Moderate Trust & 21.36 & 71.82 & 6.82\\
\hline
High Trust & 11.52 & 83.87 & 4.61\\
\hline
\multicolumn{4}{l}{\rule{0pt}{1em}\textit{Note: }}\\
\multicolumn{4}{l}{\rule{0pt}{1em}makecell[l]{Tables reflect row percentages\ Q23. How confident are you that the voting outcomes will be fair in Maricopa County, AZ?\ Q8. Generally speaking, how often can you trust other people?}}\\
\end{tabular}
\centering
\begin{tabular}[t]{l|l|l|l}
\hline
 & Q23. Fair Outcomes, AZ &  & \\
\hline
Q8 & Not\_confident & Confident & NA\_\\
\hline
Low Trust & 24.90 & 69.73 & 5.36\\
\hline
Moderate Trust & 22.61 & 73.37 & 4.02\\
\hline
High Trust & 7.23 & 85.96 & 6.81\\
\hline
\end{tabular}
\end{table}

\begin{table}
\centering
\centering
\begin{tabular}[t]{l|l|l|l}
\hline
\multicolumn{1}{l|}{Treatment} & \multicolumn{1}{l}{Control} \\
\cline{1-1} \cline{2-2}
 & Q24. Secure Voting Tech, AZ &  & \\
\hline
Q8 & Not\_confident & Confident & NA\_\\
\hline
Low Trust & 38.67 & 53.52 & 7.81\\
\hline
Moderate Trust & 28.64 & 64.55 & 6.82\\
\hline
High Trust & 17.05 & 77.88 & 5.07\\
\hline
\multicolumn{4}{l}{\rule{0pt}{1em}\textit{Note: }}\\
\multicolumn{4}{l}{\rule{0pt}{1em}makecell[l]{Tables reflect row percentages\ Q24. How confident are you that election systems in Maricopa County, AZ will be secure from hacking and other technological threats?\ Q8. Generally speaking, how often can you trust other people?}}\\
\end{tabular}
\centering
\begin{tabular}[t]{l|l|l|l}
\hline
 & Q24. Secure Voting Tech, AZ &  & \\
\hline
Q8 & Not\_confident & Confident & NA\_\\
\hline
Low Trust & 32.95 & 62.07 & 4.98\\
\hline
Moderate Trust & 25.63 & 70.35 & 4.02\\
\hline
High Trust & 9.79 & 83.83 & 6.38\\
\hline
\end{tabular}
\end{table}

\begin{table}
\centering
\centering
\begin{tabular}[t]{l|l|l|l}
\hline
\multicolumn{1}{l|}{Treatment} & \multicolumn{1}{l}{Control} \\
\cline{1-1} \cline{2-2}
 & Q25. Voter Intimidation/Violence, AZ &  & \\
\hline
Q8 & not\_concerned & concerned & NA\_\\
\hline
Low Trust & 35.94 & 55.86 & 8.20\\
\hline
Moderate Trust & 43.18 & 50.00 & 6.82\\
\hline
High Trust & 52.07 & 43.32 & 4.61\\
\hline
\multicolumn{4}{l}{\rule{0pt}{1em}\textit{Note: }}\\
\multicolumn{4}{l}{\rule{0pt}{1em}makecell[l]{Tables reflect row percentages\ Q25. Thinking about Maricopa County, AZ, how concerned should voters feel about potential violence, threats of violence, or intimidation while voting in person at their local polling place?\ Q8. Generally speaking, how often can you trust other people?}}\\
\end{tabular}
\centering
\begin{tabular}[t]{l|l|l|l}
\hline
 & Q25. Voter Intimidation/Violence, AZ &  & \\
\hline
Q8 & not\_concerned & concerned & NA\_\\
\hline
Low Trust & 43.68 & 51.34 & 4.98\\
\hline
Moderate Trust & 50.25 & 45.73 & 4.02\\
\hline
High Trust & 60.43 & 32.77 & 6.81\\
\hline
\end{tabular}
\end{table}

\begin{table}
\centering
\centering
\begin{tabular}[t]{l|l|l|l}
\hline
\multicolumn{1}{l|}{Treatment} & \multicolumn{1}{l}{Control} \\
\cline{1-1} \cline{2-2}
 & Q26. Safe In-person Voting, AZ &  & \\
\hline
Q8 & Not\_confident & Confident & NA\_\\
\hline
Low Trust & 26.95 & 64.45 & 8.59\\
\hline
Moderate Trust & 22.73 & 70.45 & 6.82\\
\hline
High Trust & 11.98 & 83.41 & 4.61\\
\hline
\multicolumn{4}{l}{\rule{0pt}{1em}\textit{Note: }}\\
\multicolumn{4}{l}{\rule{0pt}{1em}makecell[l]{Tables reflect row percentages\ Q26. How confident...are you that in person polling places in Maricopa County, AZ will be safe places for voters...in November?\ Q8. Generally speaking, how often can you trust other people?}}\\
\end{tabular}
\centering
\begin{tabular}[t]{l|l|l|l}
\hline
 & Q26. Safe In-person Voting, AZ &  & \\
\hline
Q8 & Not\_confident & Confident & NA\_\\
\hline
Low Trust & 22.61 & 72.03 & 5.36\\
\hline
Moderate Trust & 15.58 & 80.40 & 4.02\\
\hline
High Trust & 6.38 & 86.81 & 6.81\\
\hline
\end{tabular}
\end{table}

\begin{table}
\centering
\centering
\begin{tabular}[t]{l|l|l|l}
\hline
\multicolumn{1}{l|}{Treatment} & \multicolumn{1}{l}{Control} \\
\cline{1-1} \cline{2-2}
 & Q27. Election Official Approval, AZ &  & \\
\hline
Q8 & disapprove & approve & NA\_\\
\hline
Low Trust & 30.47 & 60.55 & 8.98\\
\hline
Moderate Trust & 26.36 & 66.36 & 7.27\\
\hline
High Trust & 14.75 & 80.65 & 4.61\\
\hline
\multicolumn{4}{l}{\rule{0pt}{1em}\textit{Note: }}\\
\multicolumn{4}{l}{\rule{0pt}{1em}makecell[l]{Tables reflect row percentages\ Q27. Do you approve or disapprove of the way election officials in Maricopa County, AZ are handling their jobs?\ Q8. Generally speaking, how often can you trust other people?}}\\
\end{tabular}
\centering
\begin{tabular}[t]{l|l|l|l}
\hline
 & Q27. Election Official Approval, AZ &  & \\
\hline
Q8 & disapprove & approve & NA\_\\
\hline
Low Trust & 26.44 & 68.20 & 5.36\\
\hline
Moderate Trust & 21.61 & 73.87 & 4.52\\
\hline
High Trust & 13.19 & 79.15 & 7.66\\
\hline
\end{tabular}
\end{table}

\subsection{Sample Demographics and
Balance}\label{sample-demographics-and-balance}

\begin{table}

\caption{\label{tbl-1}}

\centering{

Survey sample characteristics by treatment condition

~

Control

Treatment

Overall

(N=693)

(N=695)

(N=1388)

Age categorized into eight groups

18-24

65 (9.4\%)

59 (8.5\%)

124 (8.9\%)

25-34

120 (17.3\%)

141 (20.3\%)

261 (18.8\%)

35-44

145 (20.9\%)

128 (18.4\%)

273 (19.7\%)

45-54

118 (17.0\%)

120 (17.3\%)

238 (17.1\%)

55-64

135 (19.5\%)

123 (17.7\%)

258 (18.6\%)

65-74

80 (11.5\%)

79 (11.4\%)

159 (11.5\%)

75-84

25 (3.6\%)

39 (5.6\%)

64 (4.6\%)

85-92

5 (0.7\%)

6 (0.9\%)

11 (0.8\%)

Q59. What is your current gender?

Male

286 (41.3\%)

308 (44.3\%)

594 (42.8\%)

Female

330 (47.6\%)

328 (47.2\%)

658 (47.4\%)

Other/Refused

10 (1.4\%)

6 (0.9\%)

16 (1.2\%)

Missing

67 (9.7\%)

53 (7.6\%)

120 (8.6\%)

Q61. Primary race reported by respondent

White or Caucasian

477 (68.8\%)

496 (71.4\%)

973 (70.1\%)

Black or African American

83 (12.0\%)

79 (11.4\%)

162 (11.7\%)

American Indian

10 (1.4\%)

12 (1.7\%)

22 (1.6\%)

Asian

26 (3.8\%)

30 (4.3\%)

56 (4.0\%)

Other

30 (4.3\%)

25 (3.6\%)

55 (4.0\%)

Missing

67 (9.7\%)

53 (7.6\%)

120 (8.6\%)

Q62. Highest level of education completed

Less than high school degree

20 (2.9\%)

14 (2.0\%)

34 (2.4\%)

High school graduate (high school diploma or equivalent including GED)

154 (22.2\%)

174 (25.0\%)

328 (23.6\%)

Some college but no degree

141 (20.3\%)

141 (20.3\%)

282 (20.3\%)

Associate degree in college (2-year)

69 (10.0\%)

68 (9.8\%)

137 (9.9\%)

Bachelor's degree in college (4-year)

166 (24.0\%)

157 (22.6\%)

323 (23.3\%)

Master's degree

61 (8.8\%)

70 (10.1\%)

131 (9.4\%)

Doctoral degree

5 (0.7\%)

13 (1.9\%)

18 (1.3\%)

Professional degree (JD, MD)

10 (1.4\%)

5 (0.7\%)

15 (1.1\%)

Missing

67 (9.7\%)

53 (7.6\%)

120 (8.6\%)

Party ID 3 categories, with true Independents

Republican

261 (37.7\%)

277 (39.9\%)

538 (38.8\%)

Democrat

283 (40.8\%)

283 (40.7\%)

566 (40.8\%)

Independent

82 (11.8\%)

79 (11.4\%)

161 (11.6\%)

Missing

67 (9.7\%)

56 (8.1\%)

123 (8.9\%)

Q63. Have you ever served in the Armed Forces?

Yes

53 (7.6\%)

68 (9.8\%)

121 (8.7\%)

No

573 (82.7\%)

574 (82.6\%)

1147 (82.6\%)

Missing

67 (9.7\%)

53 (7.6\%)

120 (8.6\%)

Q64. Are you now serving in the Armed Forces?

Yes

5 (0.7\%)

10 (1.4\%)

15 (1.1\%)

No

48 (6.9\%)

58 (8.3\%)

106 (7.6\%)

Missing

640 (92.4\%)

627 (90.2\%)

1267 (91.3\%)

Q65. Member of immediate family served or is currently serving

Yes

215 (31.0\%)

223 (32.1\%)

438 (31.6\%)

No

411 (59.3\%)

419 (60.3\%)

830 (59.8\%)

Missing

67 (9.7\%)

53 (7.6\%)

120 (8.6\%)

Q66. Turnout 2020

Didn't vote

129 (18.6\%)

119 (17.1\%)

248 (17.9\%)

Unsure/Ineligible

49 (7.1\%)

53 (7.6\%)

102 (7.3\%)

Voted

448 (64.6\%)

470 (67.6\%)

918 (66.1\%)

Missing

67 (9.7\%)

53 (7.6\%)

120 (8.6\%)

Q67. Who did you vote for?

Donald Trump

214 (30.9\%)

223 (32.1\%)

437 (31.5\%)

Joe Biden

254 (36.7\%)

272 (39.1\%)

526 (37.9\%)

Other

30 (4.3\%)

29 (4.2\%)

59 (4.3\%)

Missing

195 (28.1\%)

171 (24.6\%)

366 (26.4\%)

Q68. Do you plan to vote?

Yes

501 (72.3\%)

532 (76.5\%)

1033 (74.4\%)

No

56 (8.1\%)

55 (7.9\%)

111 (8.0\%)

Undecided

69 (10.0\%)

54 (7.8\%)

123 (8.9\%)

Missing

67 (9.7\%)

54 (7.8\%)

121 (8.7\%)

Table reflects column percentages.

}

\end{table}%




\end{document}
